% Created 2014-12-15 ma 15:51
\documentclass[11pt]{article}
\usepackage[utf8]{inputenc}
\usepackage[T1]{fontenc}
\usepackage{fixltx2e}
\usepackage{graphicx}
\usepackage{longtable}
\usepackage{float}
\usepackage{wrapfig}
\usepackage{rotating}
\usepackage[normalem]{ulem}
\usepackage{amsmath}
\usepackage{textcomp}
\usepackage{marvosym}
\usepackage{wasysym}
\usepackage{amssymb}
\usepackage{hyperref}
\tolerance=1000
\author{Michiel}
\date{\today}
\title{coverletter}
\hypersetup{
  pdfkeywords={},
  pdfsubject={},
  pdfcreator={Emacs 24.4.1 (Org mode 8.2.10)}}
\begin{document}

\maketitle
\tableofcontents

Dear reader,

I am a Dutchman who has been living in Australia for the past 20 years. I am
amongst other things a pianist and ran a piano studio in Brisbane. Recently I
returned to Holland and am now looking for work as a programmer.

I am self-taught in both programming and piano. I began programming in my early teens and started playing piano at nineteen. I finished a conservatorium education in Australia in my mid twenties. Programming has been an (almost clandestine) undercurrent throughout my whole life, and after having explored music for some time now I am now elevating that undercurrent to my life's next occupation. 

I believe the future is in the web as far as apps are concerned. They should be
self-deploying, and backendless. I've written some proof of concepts and
secondary modules to progress these ideas, most in JavaScript, however now
transitioning to Clojure and ClojureScript. With tools and libs such as Docker,
CouchDB, Haproxy, Consul/Serf/CoreOS, NodeJS, Cloud API's etc, a programmer can
adopt and usurp previously separate concerns such as deployment and
feature/product development, testing and iteration.

I am keen to talk to similar minded people who are interested in building web apps using the best languages, the smartest libraries and cleverest implementations resulting in the fastest deployment and quickest iteration. I am interested in learning and applying anything to further that goal. 

I hope to bring thoroughness, ideas and architectural (and artistic) insight to a team and projects. I enjoy designing and thinking about long term design and implementation of these designs. I am working on improving my skills as a practical programmer every day since I have to live with my own creations, and keep building on them. My mantras are defry, describe and delimit:
\vspace{3mm} %3mm vertical space
\renewcommand{\labelitemi}{\textbullet}

\begin{itemize}
\item \textbf{defry}: don't ever \verb~f$%*#^g~ repeat your self!
if yes -> refactor!!
\item \textbf{describe} what you're doing!
Clear logical flow, descriptive naming, choice comments, few or no corner case handling or out of place logic, explicitly type or make clear what variables are supposed to contain, use name params instead of list etc
\item \textbf{delimit}: break up in modules, pure/independant functions, not bigger than my head per function, clear global structure/architecture
\end{itemize}

Also: libraries over frameworks, beauty over speed, minimum code over added features, 12factor apps, react over angular.

I can explain what I understand (born teacher..). When I don't understand something I can get very baffled, frustrated and feel very stupid very quickly. Then I study till I understand it or decide it's nonsense. In a team I'm the 'yes but' person. 

My coding environment: laptop running Linux (Ubuntu), i3wm, Emacs and Chrome/Firefox, my own tools. All my code is on \href{http://github.com/michieljoris}{github}.

Some more background info:

\begin{itemize}
\item I went to VWO (Science preparatory school), courses in maths, history and philosophy at Universities of Leiden and Amsterdam. Degree in piano at Brisbane Griffith Conservatorium.
\item Favorite language is lisp after working through TEOCS in Common Lisp.
\item Wrote my own server, build and module system. It does for me what most people use Express, Grunt/Gulp/Brunch and Browserify for.
\item Java and C++ put me thoroughly off OOP. Functional approach is better.
\item Studying CS curriculum when time allows (SICP will be finished one day..)
\item Working knowledge of html and css (and sass and less etc), but not overly fond of it.
\item Would love to write in and learn more about Haskell and Erlang, but also D, Rust and Julia.
\item Dream project: write Lisp operating system..
\item Wasting too much time on hackernews.
\item Major long term interest: computer systems' software reliability, robustness. Ultimately
software should be somehow self repairing, self-reliant and with built-in
redundancy, and as a whole extremely fault tolerant.
\item I even keep a \href{http://www.axion5.net}{blog}!!
\end{itemize}

Some more background info is on \href{http://careers.stackoverflow.com/michieljoris}{careers.stackoverflow} and \href{http://nl.linkedin.com/in/michieljoris/}{linkedin}.
% Emacs 24.4.1 (Org mode 8.2.10)
\end{document}
