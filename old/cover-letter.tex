%% start of file `template.tex'.
%% Copyright 2006-2013 Xavier Danaux (xdanaux@gmail.com).
%
% This work may be distributed and/or modified under the
% conditions of the LaTeX Project Public License version 1.3c,
% available at http://www.latex-project.org/lppl/.


\documentclass[11pt,a4paper,sans]{moderncv}        % possible options include font size ('10pt', '11pt' and '12pt'), paper size ('a4paper', 'letterpaper', 'a5paper', 'legalpaper', 'executivepaper' and 'landscape') and font family ('sans' and 'roman')

% moderncv themes
\moderncvstyle{banking}                            % style options are 'casual' (default), 'classic', 'oldstyle' and 'banking'
\moderncvcolor{grey}                                % color options 'blue' (default), 'orange', 'green', 'red', 'purple', 'grey' and 'black'
%\renewcommand{\familydefault}{\sfdefault}         % to set the default font; use '\sfdefault' for the default sans serif font, '\rmdefault' for the default roman one, or any tex font name
%\nopagenumbers{}                                  % uncomment to suppress automatic page numbering for CVs longer than one page

% character encoding
\usepackage[utf8]{inputenc}                       % if you are not using xelatex ou lualatex, replace by the encoding you are using
%\usepackage{CJKutf8}                              % if you need to use CJK to typeset your resume in Chinese, Japanese or Korean

% adjust the page margins
\usepackage[scale=0.75]{geometry}
%\setlength{\hintscolumnwidth}{3cm}                % if you want to change the width of the column with the dates
%\setlength{\makecvtitlenamewidth}{10cm}           % for the 'classic' style, if you want to force the width allocated to your name and avoid line breaks. be careful though, the length is normally calculated to avoid any overlap with your personal info; use this at your own typographical risks...


% personal data
\name{Michiel}{van Oosten}
%% \title{Programmer}                               % optional, remove / comment the line if not wanted
\address{Olympia plein 88}{1076AH Amsterdam}{The Netherlands}% optional, remove / comment the line if not wanted; the "postcode city" and and "country" arguments can be omitted or provided empty
\phone[mobile]{+31~06~2160~4498}                   % optional, remove / comment the line if not wanted
\phone[fixed]{+31~(020)~6659~715}                    % optional, remove / comment the line if not wanted
%% \phone[fax]{+3~(456)~789~012}                      % optional, remove / comment the line if not wanted
\email{mail@axion5.net}                               % optional, remove / comment the line if not wanted
\homepage{www.axion5.net}                         % optional, remove / comment % the line if not wanted
\social[github][]{michieljoris}
\social[linkedin][]{michieljoris}
 
%% \extrainfo{\url {github.com/michieljoris}}                 % optional, remove / comment the line if not wanted
\photo[64pt][0.4pt]{michiel.jpg}                       % optional, remove / comment the line if not wanted; '64pt' is the height the picture must be resized to, 0.4pt is the thickness of the frame around it (put it to 0pt for no frame) and 'picture' is the name of the picture file
\quote{Self taught programmer. Analytical and thorough. Forever inquisitive.}                                 % optional, remove / comment the line if not wanted
\usepackage{ragged2e}
% to show numerical labels in the bibliography (default is to show no labels); only useful if you make citations in your resume
%\makeatletter
%\renewcommand*{\bibliographyitemlabel}{\@biblabel{\arabic{enumiv}}}
%\makeatother
%\renewcommand*{\bibliographyitemlabel}{[\arabic{enumiv}]}% CONSIDER REPLACING THE ABOVE BY THIS

% bibliography with mutiple entries
%\usepackage{multibib}
%\newcites{book,misc}{{Books},{Others}}
%----------------------------------------------------------------------------------
%            content
%----------------------------------------------------------------------------------
\begin{document}
 
\setcounter{page}{2}
%-----       letter       ---------------------------------------------------------
% recipient data
\recipient{Company Recruitment team}{Company, Inc.\\123 somestreet\\some city}
\date{January 01, 1984}
\opening{Dear Sir or Madam,}
\closing{Yours faithfully,}
\enclosure[Attached]{curriculum vit\ae{}}          % use an optional argument to use a string other than "Enclosure", or redefine \enclname
\makelettertitle
\justify
I would like to ask for you to consider this cover letter somewhat more
carefully than you would normally. As you can see on the attached CV I have not
so many programming credentials to my name as you would normally see. Why it
might still be interesting for us to got get in contact is because of the
following:
\renewcommand{\labelitemi}{\textbullet}
\begin{itemize} \itemsep1pt \parskip5pt \parsep0pt
\item Despite not having a formal CS education I might still have enough coding
  chops to be of help to you. I am a self-taught programmer as I am a
  self-taught pianist. I finished a conservatorium degree without previous piano
  lessons, this still amazes me really. I have been at least as thorough, and
  still am, in educating myself in programming. I have been programming on and
  off since I was 13 years old. Piano and music was just an enormous challenge
  when I was 19 and diverted me in some sense from the direction I always was
  going to go anyway. See following point:
\item I went to VWO (Science preparatory school), then started a maths degree in
  Leiden (then philosophy, picked up piano, then history, and finally piano).
\item I have a thorough knowledge of javascript, having written it for a couple of
  years now. See my \href{http://github.com/michieljoris}{github} account.
\item Since learning of lisp it's my favorite language, however web apps are written
  in javascript. Maybe not anymore now. I'm switching to ClojureScript.
\item To prove to myself I can write javascript I wrote my own server, build and
  module system. It does for me what most people use Express, Grunt and
  Browserify for.
\item I'm familiar with and/or knowledgeable of most current
  libs/trends/frameworks/languages.
\item Not fond of OOPS, prefer functional approach, I write javascript in a
  functional style as much I can, avoiding \emph{this} wherever I can..
\item Plans and ideas for multiple apps/utilities.
\item Studying CS curriculum when time allows
\item Interested in web apps, not so much web sites, and any technology that will
  make web apps easier to make, front and backend.
\item Working knowledge of html and css (and sass and less etc), but not overly fond of it.
\item Would love the opportunity to work with any of the more modern languages such
  as Haskell, Julia, D, Rust, Go, Clojure and Scala, but sticking with
  Javascript for now for reasons of marketability.
\item Dream project: write Lisp operating system.
\item Main interest: computer systems' software reliability, robustness. Ultimately
  software should be somehow self repairing, self-reliant and with built-in
  redundancy, and as a whole extremely fault tolerant.
\item I even keep a \href{http://www.axion5.net}{blog}!!
\end{itemize}
  
I believe the future is in the web as far as apps are concerned. I'm looking for similar minded people interested in building these using the best languages, the smartest libraries and cleverest implementations resulting in fast deployment and quick iteration. I am interested in learning and applying anything to further that goal. See also my \href{http://github.com/michieljoris/cape}{cape} project. 

Interests beyond web applications are distributed programming, operating systems, compilers but foremost robustness, reliability and ease of use of computer systems. A computer shouldn't fail but it does so often.

Some more background info is on \href{http://careers.stackoverflow.com/michieljoris}{careers.stackoverflow}.

\makeletterclosing

\end{document}


%% end of file `template.tex'.
