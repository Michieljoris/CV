%% start of file `template.tex'.
%% Copyright 2006-2013 Xavier Danaux (xdanaux@gmail.com).
%
% This work may be distributed and/or modified under the
% conditions of the LaTeX Project Public License version 1.3c,
% available at http://www.latex-project.org/lppl/.


\documentclass[11pt,a4paper,sans]{moderncv}        % possible options include font size ('10pt', '11pt' and '12pt'), paper size ('a4paper', 'letterpaper', 'a5paper', 'legalpaper', 'executivepaper' and 'landscape') and font family ('sans' and 'roman')

% moderncv themes
\moderncvstyle{casual}                             % style options are 'casual' (default), 'classic', 'oldstyle' and 'banking'
\moderncvcolor{green}                               % color options 'blue' (default), 'orange', 'green', 'red', 'purple', 'grey' and 'black'
%\renewcommand{\familydefault}{\sfdefault}         % to set the default font; use '\sfdefault' for the default sans serif font, '\rmdefault' for the default roman one, or any tex font name
%\nopagenumbers{}                                  % uncomment to suppress automatic page numbering for CVs longer than one page

% character encoding
\usepackage[utf8]{inputenc}                       % if you are not using xelatex ou lualatex, replace by the encoding you are using
%\usepackage{CJKutf8}                              % if you need to use CJK to typeset your resume in Chinese, Japanese or Korean

% adjust the page margins
\usepackage[scale=0.75]{geometry}
%\setlength{\hintscolumnwidth}{3cm}                % if you want to change the width of the column with the dates
%\setlength{\makecvtitlenamewidth}{10cm}           % for the 'classic' style, if you want to force the width allocated to your name and avoid line breaks. be careful though, the length is normally calculated to avoid any overlap with your personal info; use this at your own typographical risks...


% personal data
\name{Michiel}{van Oosten}
%% \title{CV}                               % optional, remove / comment the line if not wanted
%% \title{Programmer}                               % optional, remove / comment the line if not wanted
\address{Olympia plein 88}{1076AH Amsterdam}{The Netherlands}% optional, remove / comment the line if not wanted; the "postcode city" and and "country" arguments can be omitted or provided empty
\phone[mobile]{+31~06~2160~4498}                   % optional, remove / comment the line if not wanted
\phone[fixed]{+31~(020)~6659~715}                    % optional, remove / comment the line if not wanted
%% \phone[fax]{+3~(456)~789~012}                      % optional, remove / comment the line if not wanted
\email{mail@axion5.net}                               % optional, remove / comment the line if not wanted
\homepage{www.axion5.net}                         % optional, remove / comment % the line if not wanted
\social[github][]{michieljoris}
\social[linkedin][]{michieljoris}
% personal data
\photo[70pt][0.4pt]{michiel.jpg}                       % optional, remove / comment the line if not wanted; '64pt' is the height the picture must be resized to, 0.4pt is the thickness of the frame around it (put it to 0pt for no frame) and 'picture' is the name of the picture file
\quote{Self-taught. Analytical and thorough. Forever inquisitive.}                                 % optional, remove / comment the line if not wanted

% to show numerical labels in the bibliography (default is to show no labels); only useful if you make citations in your resume
%\makeatletter
%\renewcommand*{\bibliographyitemlabel}{\@biblabel{\arabic{enumiv}}}
%\makeatother
%\renewcommand*{\bibliographyitemlabel}{[\arabic{enumiv}]}% CONSIDER REPLACING THE ABOVE BY THIS

% bibliography with mutiple entries
%\usepackage{multibib}
%\newcites{book,misc}{{Books},{Others}}
%----------------------------------------------------------------------------------
%            content
%----------------------------------------------------------------------------------
\begin{document}
%\begin{CJK*}{UTF8}{gbsn}                          % to typeset your resume in Chinese using CJK
%-----       resume       ---------------------------------------------------------
\makecvtitle

\section{Education}
\cventry{1982--1988}{VWO (Preperatory Scientific Education)}{Pax Christi College}{Druten, The Netherlands}{\textit{}}{Graduated with maths and science subjects. Beta subjects at Dutch high schools.}  % arguments 3 to 6 can be left empty
\cventry{1988--1993}{Courses in Maths, Philosophy and History}{Universities of
  Leiden and Amsterdam}{Leiden and Amsterdam, The Netherlands}{\textit{}}{Enrolled in maths degree right after high school. Changed to philosophy after a year. Took some time off then started history degree. Left in 93 for Australia to do music education.}
\cventry{1995--1998}{Diploma of Music}{Griffith University}{Brisbane, Australia}{\textit{}}{Self taught pianist from 89. Auditioned in '93. Music is difficult but one of the few worthy challenges in life I think.}
%% \cventry{year--year}{Degree}{Institution}{City}{\textit{Grade}}{Description}

%% \section{Master thesis}
%% \cvitem{title}{\emph{Title}}
%% \cvitem{supervisors}{Supervisors}
%% \cvitem{description}{Short thesis abstract}

\section{Work}
%% \subsection{Vocational}
\cventry{2013--2014}{IT Consultant}{Mamre}{Brisbane, Australia}{}{Disability organisation. Implemented databases, websites and supplied general IT support{}%
\begin{itemize}%
\item Developed two websites \url{www.mamre.org.au} and \url{www.pavetheway.org.au}
\item Technologies used: Drupal, CiviCRM, Apache, Linux, VPS (Linode).
\item Held tutorials on the use of Drupal and CivCRM.
\end{itemize}}

\cventry{2012--2014}{Developer and maintainer of website}{Firstdoor}{Brisbane, Australia}{}{Business website{}%
\begin{itemize}%
\item Built from scratch on nodejs \url{www.firstdoor.com.au}
\item Technologies used: Javascript, HTML, CSS, Nodejs, AngularJS
\item Served by custom built server: \url{github.com/michieljoris/bb-server}
\end{itemize}}

\cventry{2011--2013}{Developer of web application}{Multicap}{Brisbane, Australia}{}{Roster and time sheet management app.{}%
\begin{itemize}%
\item Built using CouchDB and SmartClient
\end{itemize}}

\cventry{2001--2013}{Piano studio}{Self-employed}{Brisbane, Australia}{}{{Self taught pianist. Started playing piano at 19. Finished music conservatorium at 1998. Some public performances.}%
\begin{itemize}%
\item Loved teaching all levels for 10+ years, giving occasional charity concerts.
\end{itemize}}

%% \cventry{year--year}{Job title}{Employer}{City}{}{Description line 1\newline{}Description line 2}
\subsection{Miscellaneous}
\cventry{2011--2014}{A number of other websites and projects}{Private contractors}{Brisbane, Australia}{}{Some built on Nodejs, some using Drupal{}%
\begin{itemize}%
\item Examples: \url{www.greenglass-terrariums.com}, \url{julangart.com}
\end{itemize}}
%% \cventry{year--year}{Job title}{Employer}{City}{}{Description}

\section{Computer Languages}
\cvitemwithcomment{Javascript}{Advanced}{Main language used so last couple of years}
\cvitemwithcomment{Lisp}{Intermediate}{Did TEOCS in Common Lisp, Scheme for SICP}
\cvitemwithcomment{Clojure(Script)}{Beginner}{But very keen to pick up..}
\cvitemwithcomment{Java}{Intermediate}{Not my favorite}
\cvitemwithcomment{C}{Intermediate}{Learned to program in C as a teenager}
\cvitemwithcomment{HTML, CSS}{Intermediate}{Know what I need}
\cvitemwithcomment{Haskell, Ruby, Coffeescript, Erlang, PHP}{Beginner}{Dabbled and/or read some books on them}
%% \cvitemwithcomment{Language 3}{Skill level}{Comment}

\section{Computer skills}
\cvdoubleitem{Databases}{CouchDB, MongoDB, MySql}{Libs}{AngularJS}
\cvdoubleitem{Editors}{Emacs, Vim, Eclipse}{CMS}{Drupal, Joomla}
\cvdoubleitem{OS}{Ubuntu, Arch, Windows, MaxOs}{Other}{Docker,Vagrant}

\section{Natural Languages}
\cvitemwithcomment{Dutch}{Fluent}{Mother tongue}
\cvitemwithcomment{English}{Fluent}{Resident of Australia since '94}
\cvitemwithcomment{German}{Intermediate}{Understand it mostly}
\cvitemwithcomment{French}{High school level}{}
%% \cvitemwithcomment{Language 3}{Skill level}{Comment}

%% \section{Interests}
%% \cvitem{Speed skating}{}
%% \cvitem{Windsurfing}{}
%% \cvitem{Sailing}{}
%% \cvitem{Classical music}{}
%% \cvitem{Literature}{}

\section{Other interests}
\cvlistitem{Speed skating, windsurfing, sailing, classical music, literature.}
%% \cvlistitem{Item 2}
%% \cvlistitem{Item 3. This item is particularly long and therefore normally spans over several lines. Did you notice the indentation when the line wraps?}

%% \section{Extra 2}
%% \cvlistdoubleitem{Item 1}{Item 4}
%% \cvlistdoubleitem{Item 2}{Item 5\cite{book1}}
%% \cvlistdoubleitem{Item 3}{Item 6. Like item 3 in the single column list before, this item is particularly long to wrap over several lines.}

%% \section{References}
%% \begin{cvcolumns}
%%   \cvcolumn{Category 1}{\begin{itemize}\item Person 1\item Person 2\item Person 3\end{itemize}}
%%   \cvcolumn{Category 2}{Amongst others:\begin{itemize}\item Person 1, and\item Person 2\end{itemize}(more upon request)}
%%   \cvcolumn[0.5]{All the rest \& some more}{\textit{That} person, and \textbf{those} also (all available upon request).}
%% \end{cvcolumns}

% Publications from a BibTeX file without multibib
%  for numerical labels: \renewcommand{\bibliographyitemlabel}{\@biblabel{\arabic{enumiv}}}% CONSIDER MERGING WITH PREAMBLE PART
%  to redefine the heading string ("Publications"): \renewcommand{\refname}{Articles}
\nocite{*}
\bibliographystyle{plain}
\bibliography{publications}                        % 'publications' is the name of a BibTeX file

% Publications from a BibTeX file using the multibib package
%\section{Publications}
%\nocitebook{book1,book2}
%\bibliographystylebook{plain}
%\bibliographybook{publications}                   % 'publications' is the name of a BibTeX file
%\nocitemisc{misc1,misc2,misc3}
%\bibliographystylemisc{plain}
%\bibliographymisc{publications}                   % 'publications' is the name of a BibTeX file

\clearpage

%\clearpage\end{CJK*}                              % if you are typesetting your resume in Chinese using CJK; the \clearpage is required for fancyhdr to work correctly with CJK, though it kills the page numbering by making \lastpage undefined
\end{document}


%% end of file `template.tex'.
